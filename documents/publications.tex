\documentclass[a4paper,oneside,10pt]{article}

\usepackage{fontspec}
\setmainfont{CMU Serif Roman}
\usepackage{libertine}

\usepackage[greek,english]{babel}

\usepackage[colorlinks = true,
            linkcolor = blue,
            urlcolor  = blue,
            citecolor = blue,
            anchorcolor = blue]{hyperref}

\useshorthands{;}
\defineshorthand{;}{?}

\usepackage{graphicx,epstopdf}
\usepackage{caption}

\usepackage{float}
\usepackage{listings}
\usepackage{parskip}
\usepackage{fancyhdr}
\usepackage[export]{adjustbox}
\setcounter{secnumdepth}{5}
\setcounter{tocdepth}{5}

\pagestyle{fancy}
\fancyhf{}
\lfoot{}
\rfoot{\thepage}

\renewcommand{\headrulewidth}{0.5pt}
\renewcommand{\footrulewidth}{0.5pt}

\usepackage{tikz}
\newcommand{\roundpic}[4][]{
  \tikz\node [circle, minimum width = #2,
    path picture = {
      \node [#1] at (path picture bounding box.center) {
        \includegraphics[width=#3]{#4}};
    }] {};}

\usepackage{multicol}

\pagestyle{fancy}
\fancyhf{}
\rhead{Publications}
\lhead{Iosif Angelidis}

\begin{document}

\subsection*{Projects}
\addcontentsline{toc}{subsection}{Projects}

\begin{itemize}

\item \begin{sloppypar}
``Jobs Data''. The largest searchable job database for Lead Generation, Labour Market Analytics, Skills Demand Data and Outplacement. Over 1 billion current and historical job postings in one central tool. Covering 10 markets across North America and Europe. Updated real-time with the latest jobs per market, deduplicated and searchable. Available via portal, API and data feed. See the ads placed by existing or potential customers or competitors in the job market. Rapidly find the right job for your candidate. Identify trends and gather job market information. \url{https://www.textkernel.com/jobfeed}

\end{sloppypar}

\item \begin{sloppypar}
``iReCirc''. iReCirc is a project participating in the EU accelerator program DigiCirc (which is about circular economies). It will develop intelligent universal solutions for the recovery of nutrients and water from food processing. \url{https://digicirc.eu/irecirc/}

\end{sloppypar}

\item \begin{sloppypar}
``PHAEDON''. The PHAEDON project aims at enhancing transparency in the agri-food sector through an innovative and state-of-the-art food traceability system which supports all the stakeholders and processes of the value chain, in all stages from production to consumption. More specifically, the project will design, implement and evaluate a fully decentralized traceability system for the agri-food sector supply chain, and specifically for the supply-chain of the Sea Buckthorn from Meteora, using Distributed Ledger Technologies, Edge Computing and IoT. The system will provide food tracking (keeping record of the product at each stage) and food tracing (reconstructing the history of the data recorded by the tracking process) functionalities. At the same time, additional functionalities will be implemented for the two main end-user categories, (the professionals who are involved in the production of the food product and the consumers) including monitoring, alerting and feedback mechanisms related to food safety. The lifecycle of PHAEDON is organized in 10 trimesters (30 months) and includes a series of technical and non-technical tasks so as to assure smooth and effective execution and progress.

\end{sloppypar}

\item \begin{sloppypar}
``3DFolllicleAI''. The objectives of the project include the investigation of the clinical possibilities of automatic 3D ultrasound imaging of the ovaries in order to increase the success rates of IVF, as well as the application of artificial intelligence to implement a referral machine. The project is implemented in the framework of the RESEARCH - CREATE - INNOVATE Action and is co-financed by the European Regional Development Fund (ERDF) of the European Union and national resources, through the Research programme Competitiveness, Entrepreneurship \& Innovation (EPANEK) (project code: T1EDK-01429).

\end{sloppypar}

\item \begin{sloppypar}
``ENDORSE''. The object of the project is the design, implementation and evaluation of an original integrated platform based on advanced and emerging Information and Communication Technologies (ICT) and playfulness and biofeedback mechanisms for the promotion of self-management of Type 1 Diabetes (T1D) and obesity. ENDORSE will focus on developing an application and service environment to raise awareness, educate, monitor, comply, promote a healthy lifestyle and support decision making by allowing target groups (children and adolescents with T1D and/or obesity) to play a leading role and not to be mere observers in their health care. \url{https://endorse.biosim.ntua.gr}

\end{sloppypar}


\item \begin{sloppypar}
``GLASS''. The vision of the EU-funded GLASS project is to place EU citizens in control of their personal information and streamline access to eGovernment services across Member States and beyond. The project is citizen-centric by design, uses state-of-the-art technology and envisages to produce a model of interaction among public administration, citizens and businesses with a strong social, societal, economic, technological and scientific impact aligned with the EU eGovernment Action Plan 2016-2020 and the EU Digital Single Market Strategy. With GLASS, public administrations will enhance their open government capabilities and significantly improve the sense of trust and confidence of citizens and businesses that will have the freedom to make personal or businesses choices without the administrative burden of the past. \url{https://cordis.europa.eu/project/id/959879}

\end{sloppypar}

\item \begin{sloppypar}
``TheFSM''. Within the global food industry, food growers, manufacturers/processors, import/exporters, distributors, retailers, and packagers, are all expected to be certified to food safety standards. This highlights the importance of food safety and certification. The EU-funded TheFSM project will develop an industrial data platform to give a digital boost to the way food certification takes place in Europe. Specifically, it will build upon state-of-art blockchain technologies to create an open and collaborative virtual environment that facilitates the exchange and connection of data between different food safety actors interested in sharing information that is critical to certification. The project will conduct extensive piloting with European providers of inspection and certification services. \url{https://cordis.europa.eu/project/id/871703}

\end{sloppypar}

\item \begin{sloppypar}
``Search and Rescue''. The project will design, implement and test through a series of large scale pilot scenarios a highly interoperable, modular open architecture platform for first responders’ capitalising on expertise and technological infrastructure from both COncORDE and IMPRESS FP7 projects. The governance model of S\&R will be designed to operate more effectively and its architectural structure will allow to easily incorporate next generation R\&D and COTS solutions which will be possibly adopted in the future disaster management systems. The Model will also support a unified vision of the EU role and will provide a common framework to assess needs and integrate responses. The framework will enable supportive approach using a wider range of decisional support features and monitoring systems and will also give to first responders an effective and unified vision of (a) the dynamic changes going on during event's lifetime and (b) the capabilities and resources currently deployed in the field. \url{https://cordis.europa.eu/project/id/882897}

\end{sloppypar}

\item \begin{sloppypar}
``FENIX''. The European Union faces several challenges caused by globalization. Both the delocalization of production plants (leading to more imported products) and the instability characterizing several industrial sectors force economies to re-think their business models and re-adapt them in a new context, where the sustainability of products and processes is more relevant. Within this overall framework, the need to think about innovative business models and industrial strategies, able to answer to these new requirements is mandatory. One chance is the exploitation of digital technologies. Another is the exploitation of secondary (and critical) resources that, currently, are wasted without any recovery. The project FENIX wants to consider both these issues and their potential at the same time, proposing something that could allow Europe to re-appropriate its pertaining position in the global market. The idea is to study innovative business models and industrial strategies (based on the circular economy paradigm) enabling the development of new product-services through the definition of novel supply chains, resulting from an unconventional mix of current ones. This could allow the easy re-use, reconfiguration and modularization of production systems, the exploitation of overcapacity and the renaissance of industrial poles all over the Europe. Furthermore, the circular economy driven business models and industrial strategies proposed by project FENIX will be demonstrated in existing pilot plants, adequately reconfigured and integrated based circular economy needs. \url{https://cordis.europa.eu/project/id/760792}

\end{sloppypar}

\item \begin{sloppypar}
``ChildRescue''. ChildRescue is an EU funded project that aspires to effectively reduce the primary period between the moment a child is reported missing and the one when it is found, and to help predict and prevent the disappearance of Children in Migration, by increasing accuracy and timeliness of publicly and privately available information, by performing evidence-based predictions on the whereabouts of children in distress, and by providing location-based audience targeting of mobile alerts. ChildRescue will be instrumental in leveraging collective creativity, resourcefulness and action for a meaningful cause: to expedite more rapid and effective prevention and resolution of missing children cases, reducing the risks for the children and reuniting them with their families. \url{https://cordis.europa.eu/project/id/780938}

\end{sloppypar}

\item Developer of the platform Nomothesi@, which functions as a portal of greek legislative text from all documents of the National Printing House. We extract legal, geospatial and textual references to entities, digitalizing this data as Semantic Web data by utilizing structured/semi-structured data with ontologies. Still contributing to the project when time allows. \url{http://legislation.di.uoa.gr/}

\item \begin{sloppypar}
\textbf{MSc Thesis} ``Named Entity Recognition and Linking in Greek Legislation''. Supervisors: Manolis Koubarakis (Professor, NKUA), Ilias Chalkidis (PhD Candidate, AUEB).
The main focus was on neural networks technologies in the area of natural language processing with the aim of extracting named entities (persons, organisations, legal references, geo-political entities, public documents and unofficial geographical landmarks), 
converting them into Semantic Web data and then linking them with other third party Semantic Web data (Greek Administrative Geography, Greek DBpedia ) using Silk. \url{https://tinyurl.com/ybkqfdpz}

\end{sloppypar}

\item \begin{sloppypar}
\textbf{Undergrad Thesis} ``Capacitated Vehicle Routing with Time Windows''. Supervisors: Alex Delis (Professor, NKUA), Panagiotis Liakos (PhD, NKUA). 
The main focus was addressing the titular problem, implement an adapted $A^{*}$ algo to visualize the shortest paths between stops of the vehicles. To that end, geospatial information was being handled with a Postgres/PostGIS database. \url{https://tinyurl.com/y7gozlkm}

\end{sloppypar}

\item \begin{sloppypar}
``Cooperative Routing and Scheduling of an Electric Vehicle Fleet Managing Dynamic Customer Requests''. A work in collaboration with Alex Delis (Professor, NKUA) and Panagiotis Liakos (PhD, NKUA). 
Development of an online algo that uses 3 distinct strategies to optimize routing and scheduling information of an electric vehicle fleet requiring recharging and compare all three. The problem addressed here is a new addition to the VRP-family of NP-hard problems. \url{https://tinyurl.com/yabjsq7r}

\end{sloppypar}

\item Working on an Android app with RESTful services offering a similar functionality to Airbnb for ``e-Commerce'' class with geospatial functionalities, Google Maps and locating addresses. \url{https://tinyurl.com/y9no2wv7}

\item Participation in a 3-member team where the representation of multiple social networks as graphs, graph quries (multi-threaded) and multiple metrics regarding communities and cliques finding were implemented (in C) for ``Software Development'' undergrad course.

\item Development of a lending library website with user roles and shopping cart functionalities for ``Human-Computer Interaction'' undergrad course. \url{https://tinyurl.com/y8oq4kb3}

\end{itemize}

\subsection*{Publications}
\addcontentsline{toc}{subsection}{Publications}

\begin{itemize}

\item \textbf{D. Ntalaperas, C. Christophoridis, \underline{I. Angelidis}, D. Iossifidis, M.-F. Touloupi, D. Vergeti and E. Politi}. ``\textit{Intelligent Tools to Monitor, Control and Predict Wastewater Reclamation and Reuse}''. Sensors 2022, 22, 3068. \url{https://tinyurl.com/5y7zf54h}

\item \textbf{\underline{I. Angelidis}, E. Politi, G. Vafeiadis, D. Vergeti, D. Ntalaperas, N. Papageorgopoulos}. ``\textit{A lightweight Ontology for real time semantic correlation of situation awareness data generated for first responders}''. The International Conference on Computational Science and Computational Intelligence (CSCI). Las Vegas, USA, 15-17 December, 2021. \url{https://tinyurl.com/5eczhabu}

\item \textbf{D. Ntalaperas, \underline{I. Angelidis}, G. Vafeiadis, D. Vergeti}. ``\textit{A Decision-Support System for the Digitization of Circular Supply Chains}''. Rosa, P., Terzi, S. (eds) New Business Models for the Reuse of Secondary Resources from WEEEs. SpringerBriefs in Applied Sciences and Technology. Springer, 19 June 2021. \url{https://tinyurl.com/3h222n28}

\item \textbf{P. Rosa (editor), S. Terzi (editor), B. Kopacek, C. Sassanelli, R. Rocca, S. Galparoli, A. Caielli, I. Birloaga, N. M. Ippolito, F. Vegliò, L. Poudelet, A. Castellví, L. Calvo, R. Ahlers, D. Ntalaperas, \underline{I. Angelidis}, G. Vafeiadis, D. Vergeti, A. Spiliotis, A. Bianchin, G. Smyrnakis}. ``\textit{New Business Models for the Reuse of Secondary Resources from WEEEs The FENIX Project: The FENIX Project}''. \url{https://tinyurl.com/u5jh6uj4}

\item \textbf{T. Beris, \underline{I. Angelidis}, I. Chalkidis, C. Nikolaou, C. Papaloukas, P. Soursos and M. Koubarakis}. ``\textit{Towards a Decentralised, Trusted, Intelligent and Linked Public Sector. A Report from the Greek Trenches}''. LDOW/LDDL workshop, The Web Conference (WWW 2019), San Francisco, CA, USA, 13 May, 2019. \url{https://tinyurl.com/ybpvtgrl}

\item \textbf{\underline{I. Angelidis}, I. Chalkidis, C. Nikolaou, P. Soursos and M. Koubarakis}. ``\textit{Nomothesia: A Linked Data Platform for Greek Legislation}''. MIREL workshop, Luxembourg Logic for AI Summit (LuxLogAI 2018), Luxembourg, 17 September, 2018. \url{https://tinyurl.com/y78m6o3q}

\item \textbf{\underline{I. Angelidis}, I. Chalkidis and M. Koubarakis}. ``\textit{Named Entity Recognition, Linking and Generation for Greek Legislation}''. The 31st international conference on Legal Knowledge and Information Systems (JURIX 2018). Groningen, The Netherlands, 12-14 December‚ 2018. \url{https://tinyurl.com/y8faqe2l}

\item \textbf{D. Punjani, K. Singh, A. Both, M. Koubarakis, \underline{I. Angelidis}, K. Bereta, T. Beris, D. Bilidas, T. Ioannidis, N. Karalis, C. Lange, D. Pantazi, C. Papaloukas and G. Stamoulis}. ``\textit{Template-Based Question Answering over Linked Geospatial Data}''. GIR 2018 workshop. Collocated with ACM SIGSPATIAL. Seattle, USA, 6 November, 2018. \url{https://tinyurl.com/y9m4n6am}

\item \textbf{P. Liakos, \underline{I. Angelidis} and A. Delis}. ``\textit{Cooperative Routing and Scheduling of an Electric Vehicle Fleet Managing Dynamic Customer Requests}''. Proceedings of the International Conference on Cooperative Information Systems (CoopIS), Rhodes, 28 October, 2016. \url{https://tinyurl.com/y7e9t4s5}

\end{itemize}

\end{document}
