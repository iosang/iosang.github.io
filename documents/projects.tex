\sectionTitle{Projects}{\faLaptop}

\begin{projects}
	\project
	{PHAEDON}{2020 - present}
	{}
	{The PHAEDON project aims at enhancing transparency in the agri-food sector through an innovative and state-of-the-art food traceability system which supports all the stakeholders and processes of the value chain, in all stages from production to consumption. More specifically, the project will design, implement and evaluate a fully decentralized traceability system for the agri-food sector supply chain, and specifically for the supply-chain of the Sea Buckthorn from Meteora, using Distributed Ledger Technologies, Edge Computing and IoT. The system will provide food tracking (keeping record of the product at each stage) and food tracing (reconstructing the history of the data recorded by the tracking process) functionalities. At the same time, additional functionalities will be implemented for the two main end-user categories, (the professionals who are involved in the production of the food product and the consumers) including monitoring, alerting and feedback mechanisms related to food safety. The lifecycle of PHAEDON is organized in 10 trimesters (30 months) and includes a series of technical and non-technical tasks so as to assure smooth and effective execution and progress.}
	{Hyperledger Fabric, Android Studio, Java, REST, Maven, IntelliJ, Swagger}

	\project
	{3DFolllicleAI}{2020 - present}
	{}
	{The objectives of the project include the investigation of the clinical possibilities of automatic 3D ultrasound imaging of the ovaries in order to increase the success rates of IVF, as well as the application of artificial intelligence to implement a referral machine. The project is implemented in the framework of the RESEARCH - CREATE - INNOVATE Action and is co-financed by the European Regional Development Fund (ERDF) of the European Union and national resources, through the Research programme Competitiveness, Entrepreneurship \& Innovation (EPANEK) (project code: T1EDK-01429).}
	{Python, Keras, TF, SciKit Learn, Numpy, Pandas, PyCharm, REST, flask, Docker}

	\project
	{ENDORSE}{2020 - present}
	{\website{https://endorse.biosim.ntua.gr}{endorse.biosim.ntua.gr}}
	{The object of the project is the design, implementation and evaluation of an original integrated platform based on advanced and emerging Information and Communication Technologies (ICT) and playfulness and biofeedback mechanisms for the promotion of self-management of Type 1 Diabetes (T1D) and obesity. ENDORSE will focus on developing an application and service environment to raise awareness, educate, monitor, comply, promote a healthy lifestyle and support decision making by allowing target groups (children and adolescents with T1D and/or obesity) to play a leading role and not to be mere observers in their health care.}
	{REST, Java, Spring, Elasticsearch, IntelliJ, Maven, PyCharm, Docker}

	\project
	{iReCirc}{2021  -present}
	{\website{https://digicirc.eu/irecirc/}{digicirc.eu/irecirc}}
	{iReCirc is a project participating in the EU accelerator program DigiCirc (which is about circular economies). It will develop intelligent universal solutions for the recovery of nutrients and water from food processing.}
	{Python, Keras, TF, SciKit Learn, Numpy, Pandas, flask, REST, HTML 5, CSS 3, Maven, IntelliJ, PyCharm, Java, Javascript, Docker, CI/CD}

	\project
	{GLASS}{2020 - present}
	{\website{https://cordis.europa.eu/project/id/959879}{cordis.europa.eu/project/id/959879}}
	{The vision of the EU-funded GLASS project is to place EU citizens in control of their personal information and streamline access to eGovernment services across Member States and beyond. The project is citizen-centric by design, uses state-of-the-art technology and envisages to produce a model of interaction among public administration, citizens and businesses with a strong social, societal, economic, technological and scientific impact aligned with the EU eGovernment Action Plan 2016-2020 and the EU Digital Single Market Strategy. With GLASS, public administrations will enhance their open government capabilities and significantly improve the sense of trust and confidence of citizens and businesses that will have the freedom to make personal or businesses choices without the administrative burden of the past.}
	{Hyperledger Fabric, Hybrid Encryption, ABAC, API Gateway, Java, jCasbin, Spring, IntelliJ, REST, Swagger, Docker}

	\project
	{TheFSM}{2020 - present}
	{\website{https://cordis.europa.eu/project/id/871703}{cordis.europa.eu/project/id/871703}}
	{Within the global food industry, food growers, manufacturers/processors, import/exporters, distributors, retailers, and packagers, are all expected to be certified to food safety standards. This highlights the importance of food safety and certification. The EU-funded TheFSM project will develop an industrial data platform to give a digital boost to the way food certification takes place in Europe. Specifically, it will build upon state-of-art blockchain technologies to create an open and collaborative virtual environment that facilitates the exchange and connection of data between different food safety actors interested in sharing information that is critical to certification. The project will conduct extensive piloting with European providers of inspection and certification services.}
	{Java, jCasbin, API Gateway, ABAC, Hybrid encryption, Thymeleaf, Spring, Javascript, IntelliJ, MongoDB, MySQL, REST, HTML 5, CSS 3, Swagger, Docker, CI/CD}

	\newpage

	\project
	{Search and Rescue}{2020 - present}
	{\website{https://cordis.europa.eu/project/id/882897}{cordis.europa.eu/project/id/882897}}
	{The project will design, implement and test through a series of large scale pilot scenarios a highly interoperable, modular open architecture platform for first responders’ capitalising on expertise and technological infrastructure from both COncORDE and IMPRESS FP7 projects. The governance model of S\&R will be designed to operate more effectively and its architectural structure will allow to easily incorporate next generation R\&D and COTS solutions which will be possibly adopted in the future disaster management systems. The Model will also support a unified vision of the EU role and will provide a common framework to assess needs and integrate responses. The framework will enable supportive approach using a wider range of decisional support features and monitoring systems and will also give to first responders an effective and unified vision of (a) the dynamic changes going on during event's lifetime and (b) the capabilities and resources currently deployed in the field.}
	{REST, RDF, OWL, Maven, Java, IntelliJ, Elasticsearch, Docker}

	\project
	{FENIX}{2020}
	{\website{https://cordis.europa.eu/project/id/760792}{cordis.europa.eu/project/id/760792}}
	{The European Union faces several challenges caused by globalization. Both the delocalization of production plants (leading to more imported products) and the instability characterizing several industrial sectors force economies to re-think their business models and re-adapt them in a new context, where the sustainability of products and processes is more relevant. Within this overall framework, the need to think about innovative business models and industrial strategies, able to answer to these new requirements is mandatory. One chance is the exploitation of digital technologies. Another is the exploitation of secondary (and critical) resources that, currently, are wasted without any recovery. The project FENIX wants to consider both these issues and their potential at the same time, proposing something that could allow Europe to re-appropriate its pertaining position in the global market. The idea is to study innovative business models and industrial strategies (based on the circular economy paradigm) enabling the development of new product-services through the definition of novel supply chains, resulting from an unconventional mix of current ones. This could allow the easy re-use, reconfiguration and modularization of production systems, the exploitation of overcapacity and the renaissance of industrial poles all over the Europe. Furthermore, the circular economy driven business models and industrial strategies proposed by project FENIX will be demonstrated in existing pilot plants, adequately reconfigured and integrated based circular economy needs.}
	{Python, Keras, TF, SciKit Learn, Numpy, Pandas, flask, REST, HTML 5, CSS 3, Java, Javascript, Maven, IntelliJ, PyCharm, Thymeleaf, Spring, Docker, CI/CD}

	\project
	{ChildRescue}{2020}
	{\website{https://cordis.europa.eu/project/id/780938}{cordis.europa.eu/project/id/780938}}
	{ChildRescue is an EU funded project that aspires to effectively reduce the primary period between the moment a child is reported missing and the one when it is found, and to help predict and prevent the disappearance of Children in Migration, by increasing accuracy and timeliness of publicly and privately available information, by performing evidence-based predictions on the whereabouts of children in distress, and by providing location-based audience targeting of mobile alerts. ChildRescue will be instrumental in leveraging collective creativity, resourcefulness and action for a meaningful cause: to expedite more rapid and effective prevention and resolution of missing children cases, reducing the risks for the children and reuniting them with their families.}
	{Android Studio, Java, Maven}

	\project
	{Nomothesi@}{2017 – 2019}
	{\website{http://legislation.di.uoa.gr/}{legislation.di.uoa.gr}}
	{Nomothesi@ is a platform which makes Greek legislation available on the Web as linked open data. It provides open access in a library of more than 12,000 pieces of legislation, interlinked with approximately 124,400 connections and over 8,000 unique entities, such as geospatial entities, persons, organizations and geographical landmarks (points of interest).}
	{Python, Keras, TF, SciKit Learn, Numpy, Pandas, (Geo/st)SPARQL, REST, HTML 5, CSS 3, RDF, OWL, Maven, PyCharm, \LaTeX}

	\project
	{Airbnb-like Android application}{2017 – 2018}
	{\link{https://www.dropbox.com/sh/46dg71devshdvv1/AAAvynY_ZJJcsGwdC6ZGsQg5a?dl=0}{\faDropbox https://tinyurl.com/y9no2wv7}}
	{Fully-functional Android application like Airbnb supporting Google Maps API, different user roles, registration and room reservation, as well as hosting rooms.}
	{Android Studio, Java, REST, Glassfish, TLS/SSL, JWT, Hibernate, XML, Gradle, HTML 5, CSS 3, Bitbucket, Netbeans, \LaTeX}

	\project
	{Copernicus App Lab}{2017 – 2018}
	{\website{https://cordis.europa.eu/project/id/730124}{cordis.europa.eu/project/id/730124}}
	{Geospatial data manipulation and production in the topics of LAI (Leaf Area Index) of Paris (from netCDF format), NO2, O3, UV emissions in Europe and oil-spills in Sweeden for the Copernicus App Lab in addition to writing of scripts to automate tasks within the group and improve efficiency.}
	{Python, RDF, OWL, PostgreSQL, PostGIS, pgAdmin, Strabon, Sextant, QGis, Pycharm, Gitlab, Apache Tomcat, \LaTeX}

	\project
	{Named Entity Recognition and Linking in Greek Legislation (MSc dissertation)}{2017-2018}
	{\website{https://pergamos.lib.uoa.gr/uoa/dl/frontend/en/browse/2766525}{https://tinyurl.com/ybkqfdpz}}
	{Entity recognition in Greek legislation texts by utilizing a named entity recognizer (NER). First work of its kind for the Greek language in such an extended form and one of the few that examines legal text. Grid search on multiple neural network architectures and combination of hyper-parameters to maximize efficiency, as well as interlinking RDF data with third-party datasets.}
	{Python, Keras, TF, SciKit Learn, Numpy, Pandas, (Geo/st)SPARQL, REST, HTML 5, CSS 3, RDF, OWL, Maven, PyCharm, \LaTeX}

	\newpage

	\project
	{Development of a lending library website with user roles and shopping cart functionalities}{2015-2016}
	{\website{http://dl104.madgik.di.uoa.gr/eamgroup56/index.php}{https://tinyurl.com/y8oq4kb3}}
	{Fully-functional lending library website that supports multiple user roles, registration, shopping cart functionalities etc.}
	{PHP, PHPStorm, XAMPP stack, Javascript, jQuery, Ajax, Bootstrap, Sweet Alert, HTML 5, CSS 3, npm, BitBucket, \LaTeX}

	\project
	{Cooperative Routing and Scheduling of an Electric Vehicle Fleet Managing Dynamic Customer Requests}{2016}
	{\link{https://bitbucket.org/Metimdjai/vrppd/src/master/}{\faBitbucket vrppd} \link{https://link.springer.com/chapter/10.1007\%2F978-3-319-48472-3_7}{\aiSpringer pub} \link{https://iosang.github.io/documents/Publications/2016/LAD-Coopis16.pdf}{\faFilePdf e-print} \link{https://dblp.uni-trier.de/rec/bib1/conf/otm/LiakosAD16.bib}{\aidblp bib}}
	{Routing and scheduling a fleet of electric vehicles seeking to satisfy dynamic pickup and delivery requests in an urban environment. Accompanied by a web application to facilitate cooperation between organizations and individuals involved in urban freight transport. Geolocation and mobile devices utilized to help manage the fleet and make timely decisions. Implementation of three heuristic recharging strategies ensuring electric vehicles can restore their energy levels efficiently.}
	{Java, IntelliJ, Leaflet JS, Javascript, HTML 5, CSS 3, Bootstrap, Redis, PostgreSQL, PostGIS, REST, XML, Maven, \LaTeX}

	\project
	{Capacitated Vehicle Routing with Time Windows}{2015}
	{\link{https://bitbucket.org/Metimdjai/vehicle-routing/src/master/}{\faBitbucket vehicle-routing} \website{https://pergamos.lib.uoa.gr/uoa/dl/frontend/en/browse/1324180}{https://tinyurl.com/y7gozlkm}}
	{Implementation to tackle the titular problem. Combination of literature algorithms and local search heuristics. A* implementation for pair-wise shortest paths. In-memory Jedis store to improve performance. Utilization of dynamic real-time traffic information. KML and HTML rendering of the generated solution.}
	{Java, IntelliJ, Leaflet JS, Javascript, HTML 5, CSS 3, Bootstrap, Redis, PostgreSQL, PostGIS, REST, XML, Maven, \LaTeX}

\end{projects}