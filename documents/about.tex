\par{
Iosif Angelidis is an IT professional with \textbf{four and a half (4.5)} years of experience in software development. His IT skills span from IT programming with actual experience in C, C++, Java and Python, among other languages. This knowledge is gained both from self-motive to invest in new skill-sets and as part of his two years engagement in a Named Entity Recognizer thesis, which recognizes and extracts persons, organizations, geo-political entities, legal references, public documents and unofficial geographical landmarks found in the Government Gazette by utilizing Neural Networks and Deep Learning technologies.}

\par{During his military tour, Iosif has been occupied in the IT Division offering his experience in supporting 3 projects. This support came in the form of major refactoring, implementation of new features, optimization and critical performance gains, as well as general user support. His responsibilities further included the training of new staff and knowledge transferring.}

\par{Working at UBITECH since July 2020, he has contributed technical implementations, partially managed and delivered reports for multiple H2020 and national projects co-funded by the EU. Notably, he has been involved in the development of a data market, Android applications, ABAC mechanisms, API Gateways, semantic models design, hybrid encryption protocols, deep learning models and RESTful services deployed in a dockerized environment. He self-invests in learning management and software architect skills as well by conducting requirement analysis, designing conceptual, technical and deployment architectures and applying Agile methodology and User stories in the analysis process. Currently, he is working as a tech lead, supervising the development of a number of projects, as well as contributing to the code development.}

\par{Last, but not least, Iosif Angelidis has shown particular interest in Artificial Intelligence courses both as a tutor (assistant professor) and as a practitioner.}
