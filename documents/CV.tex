\documentclass[a4paper,oneside,10pt]{article}

\usepackage{fontspec}
\setmainfont{CMU Serif Roman}
\usepackage{libertine}

\usepackage[greek,english]{babel}

\usepackage[colorlinks = true,
            linkcolor = blue,
            urlcolor  = blue,
            citecolor = blue,
            anchorcolor = blue]{hyperref}

\useshorthands{;}
\defineshorthand{;}{?}

\usepackage{graphicx,epstopdf}
\usepackage{caption}

\usepackage{float}
\usepackage{listings}
\usepackage{parskip}
\usepackage{fancyhdr}
\usepackage[export]{adjustbox}
\setcounter{secnumdepth}{5}
\setcounter{tocdepth}{5}

\pagestyle{fancy}
\fancyhf{}
\lfoot{}
\rfoot{\thepage}

\renewcommand{\headrulewidth}{0.5pt}
\renewcommand{\footrulewidth}{0.5pt}

\usepackage{tikz}
\newcommand{\roundpic}[4][]{
  \tikz\node [circle, minimum width = #2,
    path picture = {
      \node [#1] at (path picture bounding box.center) {
        \includegraphics[width=#3]{#4}};
    }] {};}

\usepackage{multicol}

\pagestyle{fancy}
\fancyhf{}
\rhead{Curriculum Vitae}
\lhead{Iosif Angelidis}

\begin{document}

\begin{multicols}{2} 
\roundpic{5cm}{5cm}{IMG_20190204_195706_ex.jpg}
\columnbreak

\begin{flushright}
\textlatin{Name}: \textlatin{Iosif}

\textlatin{Surname}: \textlatin{Angelidis}

\textlatin{Birth}: \textlatin{June 16, 1993}

\textlatin{Location}: \textlatin{Athens, Greece}

\textlatin{Mobile}: \textlatin{\href{tel:306947747843}{+30 694 7747 843}}

\textlatin{Phone}: \textlatin{\href{tel:302108843227}{+30 210 884 3227}}

\textlatin{e-mail}: \textlatin{\href{mailto:iosif.angelidis@gmail.com}{iosif.angelidis@gmail.com}}

\end{flushright}

\end{multicols}

\subsection*{Contact}
\addcontentsline{toc}{subsection}{Contact}

\begin{multicols}{9}
\begin{itemize}

\item[]\href{https://iosang.github.io}{\includegraphics[scale=0.08]{assets/home.pdf}}

\item[]\href{https://www.facebook.com/metimdjai}{\includegraphics[scale=0.2]{assets/facebook-original.pdf}}

\item[]\href{https://www.linkedin.com/in/iosif-angelidis/}{\includegraphics[scale=0.2]{assets/linkedin-original.pdf}}

\item[]\href{https://twitter.com/metimdjai}{\includegraphics[scale=0.2]{assets/twitter-original.pdf}}

\item[]\href{https://github.com/iosang}{\includegraphics[scale=0.2]{assets/github-original.pdf}}

\item[]\href{https://gitlab.com/metimdjai}{\includegraphics[scale=0.2]{assets/gitlab-original.pdf}}

\item[]\href{https://bitbucket.org/Metimdjai/}{\includegraphics[scale=0.2]{assets/bitbucket-original.pdf}}

\item[]\href{https://dblp.uni-trier.de/pers/hd/a/Angelidis:Iosif}{\includegraphics[scale=0.15]{assets/dblp.pdf}}

\item[]\href{https://scholar.google.gr/citations?user=r8jJLocAAAAJ&hl=en}{\includegraphics[scale=0.3]{assets/icons8-google-scholar.pdf}}

\end{itemize}
\end{multicols}

\vspace{30mm}

\begin{figure}[hbt!]
	\centering
	\includegraphics[scale=0.5]{qrcode.png}%
\end{figure}

\newpage

\subsection*{Short summary}
\addcontentsline{toc}{subsection}{Short summary}

\begin{sloppypar}
	Iosif Angelidis finished his undergraduate 
	dissertation in the area of ``Theoretical Informatics'' and completed his post-graduate studies in ``Information and Data
	Management''. He became a member of \href{http://www.madgik.di.uoa.gr}{MADgIK} and the \href{http://ai.di.uoa.gr/}{AI group} of the University of Athens. He has worked as a researcher in the same deparment, focusing on various aspects of knowledge harvesting, representation, reasoning and analytics.
\end{sloppypar}

\begin{itemize}

	\item Passionate about Java backend development and RESTful APIs.

	\item Efficient data manipulation and task automation via Python scripts.

	\item Emphasis in the documentation of the final product.

	\item Optimizing implementations when a deadline allows it.

	\item Employment of a strong theoretical background.

	\item Hard working, eager to learn and happy to assist peers.

\end{itemize}


\begin{sloppypar}
	Currently, he is seeking employment, aiming to participate in interesting projects to expand his horizons and learn new technologies and workflows. \textbf{Military obligations fulfilled by July 10, 2020.}
\end{sloppypar}

\subsection*{Working experience}
\addcontentsline{toc}{subsection}{Working experience}

\begin{itemize}

	\item 2017 - 2019: Researcher/software developer in the projects ``\href{https://cordis.europa.eu/project/id/730124}{Copernicus App Lab}'' (Horizon2020, EU Research and Innovation Programme), ``\href{http://legislation.di.uoa.gr}{Nomothesi@}'', ``Choronomothesia''. 

	\item 2015 - 2019: Teaching Assistant in the National and Kapodistrian University of Athens, Department of Informatics and Telecommunications, in: Artificial Intelligence, Object Oriented Programming Lab for Object Oriented Programming. Supervision/development of educational material for Algorithms and Complexity and Operations Research undergrad courses.

\end{itemize}

\subsection*{Skills}
\addcontentsline{toc}{subsection}{Skills}

\begin{itemize}

\item Soft Skills: communication, team player, patient, ability to cooperate remotely.

\item Programming Languages: \textlatin{C, C++, Java, Python}.

\item Operating Systems: \textlatin{GNU/LINUX (including advanced distros like Arch and Manjaro), Microsoft Windows, MacOS}.

\item Shell Scripting: \textlatin{bash, zsh}.

\item Frameworks \& Libraries: \textlatin{Spring ecosystem, Bootstrap, Keras, Tensorflow, SciKit Learn, Numpy, Pandas}.

\item Style Sheet and Doc Languages: \textlatin{HTML5, CSS3, XML, SOAP, RDF, OWL}.

\item Servers \& Databases: \textlatin{Apache Tomcat, Wildfly/Jboss, Glassfish, Apache HTTP Server/XAMPP, PHPMyAdmin, MySQL Workbench, pgAdmin, SQL, Oracle, PostgreSQL/PostGIS, Redis, (Geo/st)SPARQL}.

\item IDEs: \textlatin{Intellij, Android Studio, PyCharm, PHPStorm, VS Code, Netbeans}.

\item Geospatial Systems: \textlatin{Sextant, Strabon, QGis}.

\item Version Control Systems: \textlatin{Mercurial, Git, CVS, GitHub, GitLab, Bitbucket}.

\item General: \textlatin{AOP, design patterns, \LaTeX, Microsoft Office Suite, Matlab, Gimp}.

\end{itemize}

\subsection*{Projects}
\addcontentsline{toc}{subsection}{Projects}

\begin{itemize}

\item Developer of the platform \href{http://legislation.di.uoa.gr/}{Nomothesi@}, which functions as a portal of greek legislative text from all documents of the National Printing House. We extract legal, geospatial and textual references to entities, digitalizing this data as Semantic Web data by utilizing structured/semi-structured data with ontologies. Still contributing to the project when time allows.

\item \begin{sloppypar}
\textbf{MSc Thesis} ``\href{https://pergamos.lib.uoa.gr/uoa/dl/frontend/en/browse/2766525}{Named Entity Recognition and Linking in Greek Legislation}''. Supervisors: Manolis Koubarakis (Professor, NKUA), Ilias Chalkidis (PhD Candidate, AUEB).
The main focus was on neural networks technologies in the area of natural language processing with the aim of extracting named entities (persons, organisations, legal references, geo-political entities, public documents and unofficial geographical landmarks), 
converting them into Semantic Web data and then linking them with other third party Semantic Web data (Greek Administrative Geography, Greek DBpedia ) using Silk.

\end{sloppypar}

\item \begin{sloppypar}
\textbf{Undergrad Thesis} ``\href{http://efessos.lib.uoa.gr/applications/disserts.nsf/0f1ab5fee83fbb88c225770c0042ce4f/8da6d56136caaacec2257ea6004c9349?OpenDocument}{Capacitated Vehicle Routing with Time Windows}''. Supervisors: Alex Delis (Professor, NKUA), Panagiotis Liakos (PhD, NKUA). 
The main focus was addressing the titular problem, implement an adapted $A^{*}$ program to visualize the shortest paths between stops of the vehicles. To that end, geospatial information was being handled with a Postgres database with PostGIS functionality.

\end{sloppypar}

\item \begin{sloppypar}
``\href{https://bitbucket.org/Metimdjai/vrppd/src/master/}{Cooperative Routing and Scheduling of an Electric Vehicle Fleet Managing Dynamic Customer Requests}''. A work in collaboration with Alex Delis (Professor, NKUA) and Panagiotis Liakos (PhD, NKUA). 
The goal of this project was to develop an online algorithm that uses 3 distinct strategies to optimize routing and scheduling information of an electric vehicle fleet (which requests recharging) and compare all three. The problem addressed here is a new addition to the VRP-family of NP-hard problems.

\end{sloppypar}

\item Working on an \href{https://www.dropbox.com/sh/46dg71devshdvv1/AAAvynY_ZJJcsGwdC6ZGsQg5a?dl=0}{Android app} with RESTful services offering a similar functionality to Airbnb for ``e-Commerce'' class with geospatial functionalities, Google Maps and locating addresses.

\item Participation in a 3-member team where the representation of multiple social networks as graphs, graph quries (multi-threaded) and multiple metrics regarding communities and cliques finding were implemented (in C) for ``Software Development'' undergrad course.

\item Development of a \href{http://dl104.madgik.di.uoa.gr/eamgroup56/index.php}{lending library website} with user roles and shopping cart functionalities for ``Human-Computer Interaction'' undergrad course.

\end{itemize}

\subsection*{Education}
\addcontentsline{toc}{subsection}{Education}

\begin{itemize}

\item 2016-2018: MSc degree in Information and Data Management, National and Kapodistrian University of Athens, Department of Informatics and Telecommunications. Grade: 9.39/10.0.

\item 2011-2016: Ptychion (4-year degree with thesis) in Computer Science - Theoretical Informatics, National and Kapodistrian University of Athens, Department of Informatics and Telecommunications. Grade: 9.34/10.0.

\item 2007: \textlatin{ECDL (Syllabus Version 4.0)} certificate in \textlatin{Databases, Concepts of IT, Word Processing, Presentations, Information and Communication, Spreadsheets, Using the Computer and Managing Files}.

\end{itemize}

\subsection*{Languages}
\addcontentsline{toc}{subsection}{Languages}

\begin{itemize}

\item English: \textlatin{Certificate of Proficiency in English, 2009 University of Michigan, First Certificate in English (Grade B), 2008 University of Cambridge.}

\item Greek: Native language.

\end{itemize}

\subsection*{Distinctions}
\addcontentsline{toc}{subsection}{Distinctions}

\begin{itemize}

\item 2011: Greek State Scholarships Foundation (IKY) scholarship for academic performance, 2011-2012.

\item 2002: second place certificate in backstroke 33m, Panellinios AC.

\end{itemize}

\subsection*{Publications}
\addcontentsline{toc}{subsection}{Publications}

\begin{itemize}

\item \textbf{T. Beris, \underline{I. Angelidis}, I. Chalkidis, C. Nikolaou, C. Papaloukas, P. Soursos and M. Koubarakis}. ``\textit{Towards a Decentralised, Trusted, Intelligent and Linked Public Sector. A Report from the Greek Trenches}''. LDOW/LDDL workshop, The Web Conference (WWW 2019), San Francisco, CA, USA, 13 May, 2019. [\href{https://dl.acm.org/citation.cfm?doid=3308560.3317077}{PUB}][\href{https://iosang.github.io/documents/Publications/2019/www19companion-206.pdf}{E-PRINT}][\href{https://iosang.github.io/documents/Publications/2019/3317077.bib}{BIB}].

\item \textbf{\underline{I. Angelidis}, I. Chalkidis, C. Nikolaou, P. Soursos and M. Koubarakis}. ``\textit{Nomothesia: A Linked Data Platform for Greek Legislation}''. MIREL workshop, Luxembourg Logic for AI Summit (LuxLogAI 2018), Luxembourg, 17 September, 2018. [\href{https://ora.ox.ac.uk/objects/uuid:b19c1428-49db-402b-8afd-b8cf588e147d}{PUB}][\href{https://iosang.github.io/documents/Publications/2018/nomothesia-linked-data.pdf}{E-PRINT}][\href{https://ora.ox.ac.uk/objects/uuid:b19c1428-49db-402b-8afd-b8cf588e147d/export_record.bibtex}{BIB}].

\item \textbf{\underline{I. Angelidis}, I. Chalkidis and M. Koubarakis}. ``\textit{Named Entity Recognition, Linking and Generation for Greek Legislation}''. The 31st international conference on Legal Knowledge and Information Systems (JURIX 2018). Groningen, The Netherlands, 12-14 December‚ 2018. [\href{https://doi.org/10.3233/978-1-61499-935-5-1}{PUB}][\href{https://iosang.github.io/documents/Publications/2018/jurix2018.pdf}{E-PRINT}][\href{https://dblp.uni-trier.de/rec/bib1/conf/jurix/AngelidisCK18.bib}{BIB}].

\item \textbf{D. Punjani, K. Singh, A. Both, M. Koubarakis, \underline{I. Angelidis}, K. Bereta, T. Beris, D. Bilidas, T. Ioannidis, N. Karalis, C. Lange, D. Pantazi, C. Papaloukas and G. Stamoulis}. ``\textit{Template-Based Question Answering over Linked Geospatial Data}''. GIR 2018 workshop. Collocated with ACM SIGSPATIAL. Seattle, USA, 6 November, 2018.[\href{https://doi.org/10.1145/3281354.3281362}{PUB}][\href{https://iosang.github.io/documents/Publications/2018/template-based-GeoQA.pdf}{E-PRINT}][\href{https://dl.acm.org/downformats.cfm?id=3281362&parent_id=3281354&expformat=bibtex}{BIB}].

\item \textbf{P. Liakos, \underline{I. Angelidis} and A. Delis}. ``\textit{Cooperative Routing and Scheduling of an Electric Vehicle Fleet Managing Dynamic Customer Requests}''. Proceedings of the International Conference on Cooperative Information Systems (CoopIS), Rhodes, 28 October, 2016. [\href{https://link.springer.com/chapter/10.1007%2F978-3-319-48472-3_7}{PUB}][\href{https://iosang.github.io/documents/Publications/2016/LAD-Coopis16.pdf}{E-PRINT}][\href{https://dblp.uni-trier.de/rec/bib1/conf/otm/LiakosAD16.bib}{BIB}].

\end{itemize}

\subsection*{Research interests}
\addcontentsline{toc}{subsection}{Research interests}

My research interests focus on Artificial Intelligence, Deep Learning, NLP and Semantic Web Technologies.

\subsection*{Hobbies and interests}
\addcontentsline{toc}{subsection}{Hobbies and interests}

\begin{itemize}

	\item Exploring the open source community, learning new technologies.

	\item Travelling, visiting archaeological sites, gaming, swimming.
	
	\item Psychology, philosophy, mathematics, history and physics.

\end{itemize}

\end{document}
